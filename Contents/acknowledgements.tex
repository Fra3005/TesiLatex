Anche se si trovano all'inizio, queste pagine di ringraziamento ho scelto di scriverle per ultime, come giusto epilogo di un percorso durato tre anni.
\\
Un cammino fatto di sfide, traguardi e profondi cambiamenti. Tra un esame e l’altro, si sono chiusi capitoli della mia vita che ormai davo per scontati, e se ne sono aperti altri che mi hanno permesso di ritrovare un senso di normalità, di equilibrio.
In questo percorso, oltre ad aver scoperto e approfondito un mondo completamente nuovo, ho incontrato persone che hanno lasciato un segno nella mia vita e che desidero ora ringraziare.
\\\\
Il mio primo e più sentito ringraziamento va alla Professoressa \textbf{Claudia d’Amato}. Non solo la migliore docente che io abbia mai avuto, ma anche un'autentica guida e un esempio di integrità e serietà a cui chiunque dovrebbe ispirarsi. È stata per me un mentore prezioso, e il suo supporto ha avuto un ruolo fondamentale in questo percorso.
\\
Un ringraziamento speciale va anche al Dott. \textbf{Roberto Basile}, che ho conosciuto prima come collega e poi come co-relatore. La sua disponibilità e la sua competenza sono state per me un grande valore aggiunto.
\\\\
Un grazie profondo va alla mia famiglia, che mi ha sostenuto in ogni momento, non solo dal punto di vista economico, ma soprattutto con l'amore e la dedizione che solo dei genitori sanno dare. Anche se non lo dico spesso, questo è il mio modo per dirvi che vi voglio bene, e che non avrei potuto desiderare genitori migliori.
Grazie a mia sorella \textbf{Valentina} e a mio cognato \textbf{Lucas}, che per me rappresentano l'essenza dell'Amore.
Dalla loro unione sono nate le mie due splendide nipotine, \textbf{Arya e Zoe}, che con il loro affetto riescono ogni volta a farmi sentire a casa, come se non fossi mai andato via.
\\\\
Grazie ai miei amici di sempre, \textbf{Carmine} e \textbf{Mattia}, che nonostante la distanza degli ultimi mesi, non mi hanno mai fatto mancare la loro presenza, vi voglio bene.
Grazie anche a \textbf{Giuseppe} e \textbf{Nicola}, con cui oggi ci vediamo meno, ma che restano per me dei punti fermi ogni volta che ci incontriamo.
\\\\
Ai miei compagni di università, \textbf{Nicolas}, \textbf{Dani}, \textbf{Gianfederico} e \textbf{Donato}, va un grazie sincero: con voi ho condiviso quelli che considero tra i migliori anni della mia vita. Un ringraziamento speciale va a Donato, con cui ho condiviso non solo l'intero percorso universitario, ma anche momenti di vita che vanno ben oltre lo studio. Senza di lui, sinceramente, non so se ce l'avrei fatta.
\\\\
Grazie ai miei nuovi amici, i \textbf{Liofilizzati}: fin da subito mi avete fatto sentire a mio agio,e anche se forse non lo sapete, mi avete aiutato a superare un periodo difficile.
\\\\
Un ringraziamento speciale va anche ai miei coinquilini, \textbf{Nicolò} e \textbf{Mauro}, che sin dal primo giorno mi hanno fatto sentire a casa. Con loro condivido ogni giorno momenti di leggerezza, risate e complicità, che hanno reso questa esperienza ancora più bella e umana.
\\\\
Desidero ringraziare tutte le persone che sono passate dalla mia vita in questi anni, anche se solo per un breve tratto: ognuna di loro ha contribuito a formare la persona che sono oggi.
\\
Un pensiero va anche a chi è appena entrato nella mia vita e ogni giorno mi incoraggia e mi spinge a dare il meglio.
\\\\
Infine, voglio ringraziare te, \textbf{nonno}. Anche se ormai non ci sei più, la tua assenza ha accompagnato silenziosamente ogni giorno di questi anni.\\
Ricordo bene quando decisi di iscrivermi alla magistrale: molti erano scettici, pensavano che, avendo già iniziato a lavorare, fosse il momento di costruire qualcosa di diverso.\\
Ma tu no. Tu sei stato l'unico a credere davvero in me, l'unico a incoraggiarmi a proseguire, a non mollare.\\
E risuona ancora nella mia mente quella tua frase che mi dicevi ogni volta che partivo:\\
\textit{Mi raccomando, sempre in gamba.}
\\\\
È anche grazie a quelle parole, semplici ma cariche di fiducia, se oggi sono arrivato fin qui.
\\
Questa tesi, nonno, è per te.
\vspace*{\fill}
\begin{flushright}
\textit{Francesco Didio}
\end{flushright}