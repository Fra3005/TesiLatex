%----------------------------------------------------------------------------------------
%Capitolo 6 - Conclusioni e sviluppi futuri. Cosa avete raggiunto con la vostra tesi? Cose c’è ancora da raggiungere?
%----------------------------------------------------------------------------------------
To conclude, in this chapter we present our final thoughts on the work carried out, with a particular focus on possible future developments that could further expand and refine the analyses conducted.
\section{Conclusions}
What inspired this master's thesis, as well as my bachelor's thesis, was the desire to use artificial intelligence to serve citizens, with the aim of fully exploiting its potential to improve quality of life and access to information. This approach guided the entire research process, orienting methodological choices and areas of investigation.
As discussed in Section \ref{sec:risks}, Artificial Intelligence today represents a technology capable of generating important benefits in numerous areas of society. However, at the same time, it also involves non-negligible risks, especially in relation to improper or unknowing use by users.
For this reason, it is essential to promote the development and adoption of safer, more reliable and transparent systems that can mitigate these risks.
In particular, \textit{Prompt engineering} has proven to be an effective tool for guiding LLMs towards more legally informed responses, thus contributing to a more responsible use of generative AI.
The results that emerged from this thesis, although preliminary, can be a starting point for further reflection and future developments, aimed at designing linguistic technologies that are increasingly ethical and at the service of the user.
\section{Future works}
To further strengthen the results that emerged from this thesis work, there are several areas that could be explored in future developments. First of all, the adoption of the \textbf{Retrieval-Augmented Generation (RAG)} prompt engineering technique, as discussed in \ref{sec: taxonomies}, could be a particularly advantageous choice when the goal is to obtain detailed and specific answers regarding a specific area.
A comparison between the RAG prompts and those explored in this study could prove interesting: in fact, it would be possible to apply RAG to a subset of questions and compare the precision and accuracy of the answers obtained using this technique with those generated using the prompts analysed in this study.
A further development could involve the integration of legal experts in the evaluation process. In this scenario, the experts, following a predefined structure, could answer a set of questions, and their answers could be compared with those generated by large linguistic models. This comparison would make it possible to assess the extent to which LLMs are able to replace the work of legal experts, at least for less complex tasks, thus contributing to a broader reflection on the automation of legal processes.